% !TEX root = presentation.tex
% { % all template changes are local to this group.
%     \begin{frame}[plain]
%         \begin{tikzpicture}[remember picture,overlay]
%             \node[at=(current page.center)] {
%                 \includegraphics[width=\paperwidth]{./img/03_tomb-raider-underworld}
%             };
%         \end{tikzpicture}
%      \end{frame}
% }


	\begin{frame}\frametitle{Hardware - Pipelines}
		\begin{columns}
			\begin{column}{0.9\textwidth}
				\begin{center}
					\includegraphics[width=\textwidth]{img/3_pipeline/pipelineDifferences_oldOpenGL.png}
					\small{2001}
				\end{center}	
			\end{column}
		\end{columns}
		\pause
		\vspace{0.8cm}
		\begin{columns}
			\begin{column}{0.9\textwidth}
				\begin{center}
					\includegraphics[width=\textwidth]{img/3_pipeline/pipelineDifferences_newOpenGL.png}
					\small{2015}
				\end{center}	
			\end{column}
		\end{columns}
		\note<1>{\textbf{[Name]} Great part of the paper stresses the point that it can easily be implemented as a preprocessing step (CPU.\\ 2001 pipeline (OpenGl 1.3)}
		\note<2>{\textbf{[Name]} 2015 we have OpenGL 4.5 with more programmable shaders and the whole process can be done on the GPU. Since PN triangles only usses the primitive, no neighboring primitives, easy in shaders.}
	\end{frame}

% \begin{frame}
% 	\frametitle{Hardware - Pipelines}
%     \todo[inline]{Hoe zou je het nu kunnen implementeren? Plus pipeline}
% \end{frame}