% !TEX root = presentation.tex
	\begin{frame}\frametitle{Cubic B\'ezier triangles}
		\begin{columns}
			\begin{column}{\textwidth}
				\begin{equation*}
					b: \mathbb{R}^2 \rightarrow \mathbb{R}^3 \text{, for } w = 1 - u - v, u, v \text{, } w \geq 0
				\end{equation*}
				\begin{equation*}
					\begin{aligned}
						b(u,v) & = \sum\limits_{i+j+k=3} b_{ijk} \frac{3!}{i!j!k!} u^i v^j w^k\\
						& = b_{300} w^3 + b_{030} u^3 + b_{003} v^3 \\
						& + b_{210} 3 w^2 u + b_{120} 3 w u^2 + b_{201} 3 w^2 v\\
						& + b_{021} 3 u^2 v + b_{102} 3 w v^2 + b_{012} 3 u v^2\\
						& + b_{111} 6 w u v.
 					\end{aligned}
				\end{equation*}
			\end{column}
		\end{columns}
	\end{frame}

	\begin{frame}\frametitle{Geometry}
		\begin{columns}
			\begin{column}{0.6\textwidth}
				\begin{center}
					\includegraphics[width=\textwidth]{img/1_single/inputPrimitive.png}
					\small{Input primitive}
				\end{center}
			\end{column}
		\end{columns}
	\end{frame}

	% Geometry process
	\begin{frame}\frametitle{Geometry - step 1}
		\begin{columns}
			\begin{column}{0.6\textwidth}
				\begin{center}
					\includegraphics[width=\textwidth]{img/1_single/geometry_1.png}
					\small{Control net}
				\end{center}
			\end{column}
			\uncover<2>{
				\begin{column}{0.4\textwidth}
					\begin{equation*}
						b_{ijk} = (iP_1 + jP_2 + kP_3)/3
					\end{equation*}

					\begin{equation*}
						\begin{aligned}
						b_{300} & = P_1,\\
						b_{030} & = P_2,\\
						b_{003} & = P_3
						\end{aligned}
					\end{equation*}
				\end{column}
			}
		\end{columns}
	\end{frame}

	\begin{frame}\frametitle{Geometry - step 2}
		\begin{columns}
			\begin{column}{0.6\textwidth}
				\begin{center}
				\includegraphics[width=\textwidth]{img/1_single/geometry_2.png}
				\small{Normal projection}
				\end{center}
			\end{column}
			\uncover<2>{
				\begin{column}{0.4\textwidth}
					\begin{equation*}
						\begin{aligned}
						w_{ij} & = (P_j - P_i) \cdot N_i \in \mathbb{R} \\
						b_{210} & = (2 P_1 + P_2 - w_{12}N1) / 3,\\
						b_{120} & = (2 P_2 + P_1 - w_{21}N2) / 3,\\
						b_{021} & = (2 P_2 + P_3 - w_{23}N2) / 3,\\
						b_{012} & = (2 P_3 + P_2 - w_{32}N3) / 3,\\
						b_{102} & = (2 P_3 + P_1 - w_{31}N3) / 3,\\
						b_{201} & = (2 P_1 + P_3 - w_{13}N1) / 3\\
						\end{aligned}
					\end{equation*}
				\end{column}
			}
		\end{columns}
	\end{frame}

	\begin{frame}\frametitle{Geometry - step 3}
		\begin{columns}
			\begin{column}{0.6\textwidth}
				\begin{center}
				\includegraphics[width=\textwidth]{img/1_single/geometry_3.png}
				\small{Center control point}
				\end{center}
			\end{column}
			\uncover<2>{
				\begin{column}{0.4\textwidth}
					\begin{equation*}
						\begin{aligned}
						E & = (b_{210} + b_{120} + b_{021} \\ 
						+ & b_{012} + b_{102} + b_{201}) / 6,\\
						V & = (P_1 + P_2 + P_3)/ 3, \\
						b_{111} & = E + (E - V) / 2
						\end{aligned}
					\end{equation*}
				\end{column}
			}
		\end{columns}
	\end{frame}	

	\begin{frame}\frametitle{Geometry - result}
		% \todo[inline]{How do you create the PN triangle geometrically}
		\begin{columns}
			\begin{column}{0.6\textwidth}
				\begin{center}
				\includegraphics[width=\textwidth]{img/1_single/geometry_4.png}
				\end{center}	
			\end{column}
		\end{columns}
	\end{frame}

		\begin{frame}\frametitle{Normals}
		\begin{columns}
			\begin{column}{0.6\textwidth}
				\begin{center}
					\includegraphics[width=\textwidth]{img/1_single/inputPrimitive.png}
					\small{Input primitive}
				\end{center}
			\end{column}
		\end{columns}
	\end{frame}

	\begin{frame}
		\frametitle{Normals}
		\begin{columns}
			\begin{column}{0.6\textwidth}
				\begin{center}
					\includegraphics[width=\textwidth]{img/1_single/normals.png}
				\end{center}	
			\end{column}
			\uncover<2>{
				\begin{column}{0.4\textwidth}
					\begin{equation*}
						A^2 + B^2 = C^2
					\end{equation*}
				\end{column}
			}
		\end{columns}
	\end{frame}

	\begin{frame}\frametitle{Normals - theory}
		\begin{columns}
			\begin{column}{0.6\textwidth}
				\begin{center}
					\includegraphics[width=\textwidth]{img/1_single/linearVsQuadraticNormals_linear.png}
					\small{Linear}
				\end{center}	
			\end{column}
		\end{columns}
		\pause
		\vspace{0.8cm}
		\begin{columns}
			\begin{column}{0.6\textwidth}
				\begin{center}
					\includegraphics[width=\textwidth]{img/1_single/linearVsQuadraticNormals_quadratic.png}
					\small{Quadratic}
				\end{center}	
			\end{column}
		\end{columns}
	\end{frame}

	\begin{frame}\frametitle{Normals - theory}
		\begin{columns}
			\begin{column}{0.6\textwidth}
				\begin{center}
					\includegraphics[width=\textwidth]{img/1_single/computingNormals.png}
				\end{center}
			\end{column}
			\uncover<2>{
				\begin{column}{0.4\textwidth}
					\begin{equation*}
						A^2 + B^2 = C^2
					\end{equation*}
				\end{column}
			}
		\end{columns}
	\end{frame}

	\begin{frame}\frametitle{Normals - result}
		\begin{columns}
			\begin{column}{0.4\textwidth}
			\begin{center}
					\includegraphics[width=\textwidth]{img/1_single/linearlyVaryingNormals.png}
					\small{Linear}
				\end{center}	
			\end{column}
			\begin{column}{0.4\textwidth}
			\begin{center}
					\includegraphics[width=\textwidth]{img/1_single/quadriticallyVaryingNormals.png}
					\small{Quadratic}
				\end{center}	
			\end{column}
		\end{columns}
	\end{frame}

	\begin{frame}
		\frametitle{Level Of Detail}
		\todo[inline]{Barycentric coordinates recap}
		\begin{equation*}
			n: \mathbb{R}^2 \rightarrow \mathbb{R}^3 \text{, for } w = 1 - u - v \text{, } u, v, w \geq 0 
		\end{equation*}
		\begin{equation*}
			\begin{aligned}
			n(u,v) & = \sum\limits_{i+j+k=2} n_{ijk} u^i v^j w^k\\
			& = n_{200} w^2 + n_{020} u^2 + n_{002} v^2\\
			& + n_{110} w u + n_{011} u v + n_{101} w v
			\end{aligned}
		\end{equation*}
	\end{frame}	

	\begin{frame}
		\frametitle{Level Of Detail}
		\todo[inline]{LOD verhaal}
		\begin{columns}
			\begin{column}[b]{0.22\textwidth}
				\begin{center}
					\includegraphics[width=\textwidth]{./img/1_single/lod_lod0.png}
					\small{0}
				\end{center}	
			\end{column}
			\begin{column}[b]{0.22\textwidth}
				\begin{center}
					\includegraphics[width=\textwidth]{./img/1_single/lod_lod1.png}	
					\small{1}
				\end{center}	
			\end{column}
			\begin{column}[b]{0.22\textwidth}
				\begin{center}
					\includegraphics[width=\textwidth]{./img/1_single/lod_lod2.png}	
					\small{2}
				\end{center}	
			\end{column}
			\begin{column}[b]{0.22\textwidth}
				\begin{center}
					\includegraphics[width=\textwidth]{./img/1_single/lod_lod3.png}	
					\small{3}
				\end{center}	
			\end{column}
		\end{columns}
	\end{frame}	

	\begin{frame}\frametitle{Construction}
		\todo[inline]{The steps. Recap of everything construct geometry and normals and evaluate less (low lod) or more points (high lod)}
		\begin{columns}
			\begin{column}{0.6\textwidth}
				\begin{center}
					% \includegraphics[width=\textwidth]{recap.png}
					\missingfigure{Recap}
				\end{center}		
			\end{column}
		\end{columns}
	\end{frame}	