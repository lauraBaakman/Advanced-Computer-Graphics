% !TEX root = presentation.tex
%!TEX root = presentation.tex
\begin{frame}\frametitle{Overview}
	\only<1>{
		\begin{center}
			\includegraphics[width=\textwidth]{./img/1_single/recap_overview.png}
		\end{center}
	}
	\only<2>{
		\begin{center}
			\includegraphics[width=\textwidth]{./img/1_single/recap_inputToGeom.png}
		\end{center}
	}
	\note<1>{\textbf{[Laura]} PN triangle is defined by geometry and normal component.}
	\note<2>{\textbf{Laura} From geometric component of input primitive to geometric component of PN triangle}
\end{frame}	

\begin{frame}\frametitle{Geometry}
	\enhancement{emphasize vertices better}
	\begin{columns}
		\begin{column}{0.6\textwidth}
			\begin{center}
				\includegraphics[width=\textwidth]{img/1_single/inputPrimitive_emphGeometry.png}
				\small{input primitive}
			\end{center}
		\end{column}
	\end{columns}
	\note[item]{\textbf{[Laura]} This a standard triangle primitive, defined by its vertices and normals.}
	\note[item]{\textbf{[Laura]} Note that we only have this input primitive, without information about its neighbors.}
\end{frame}

% Geometry process
\begin{frame}\frametitle{Geometry - Vertex Coefficients}
	\begin{columns}
		\begin{column}{0.6\textwidth}
			\begin{center}
				\includegraphics[width=\textwidth]{img/1_single/geometry_1.png}
				\small{control net}
			\end{center}
		\end{column}
			\begin{column}{0.4\textwidth}
				\only<2,3>{
				\begin{equation*}
					b_{ijk} = (iP_1 + jP_2 + kP_3)/3
				\end{equation*}
				}
				\only<3>{
				\begin{equation*}
					\begin{aligned}
					b_{300} & = P_1,\\
					b_{030} & = P_2,\\
					b_{003} & = P_3
					\end{aligned}
				\end{equation*}
				}
			\end{column}
	\end{columns}
	\note[item]<1>{\textbf{[Laura]} This is the topology of the final patch}
	\note[item]<1>{\textbf{[Laura]} Mention which vertices are vertex, tangent and center coefficient.}	
	\note[item]<2>{\textbf{[Laura]} These are all the initial control point. Evenly divided on the triangle. -> formula}
	\note[item]<2>{\textbf{[Laura]} Nice formula}
	\note[item]<3>{\textbf{[Laura]} Stress that the vertex coefficients/control points are the one on the original vertices and that they do not move.}
\end{frame}

\begin{frame}\frametitle{Geometry - Tangent Coefficients}
	\begin{columns}
		\begin{column}{0.6\textwidth}
			\begin{center}
			\includegraphics[width=\textwidth]{img/1_single/geometry_2.png}
			\small{normal projection}
			\end{center}
		\end{column}
		\uncover<2>{
			\begin{column}{0.4\textwidth}
				\begin{equation*}
					\begin{aligned}
					w_{ij} & = (P_j - P_i) \cdot N_i \in \mathbb{R} \\
					b_{210} & = \frac{2 P_1 + P_2 - w_{12}N_1}{3},\\
					~ & \vdots \\
					b_{201} & = \frac{2 P_1 + P_3 - w_{13}N_1}{3}
					\end{aligned}
				\end{equation*}
			\end{column}
		}
	\end{columns}
	\note{\textbf{[Laura]} Define a plane using the closest vertex and its normal. Find the point on this plane that is closest to the uniformly distributed point.}
\end{frame}

% b_{120} & = (2 P_2 + P_1 - w_{21}N2) / 3,\\
% b_{021} & = (2 P_2 + P_3 - w_{23}N2) / 3,\\
% b_{012} & = (2 P_3 + P_2 - w_{32}N3) / 3,\\
% b_{102} & = (2 P_3 + P_1 - w_{31}N3) / 3,\\

\begin{frame}\frametitle{Geometry - Center Coefficient}
	\begin{columns}
		\begin{column}{0.6\textwidth}
			\begin{center}
			\includegraphics[width=\textwidth]{img/1_single/geometry_3.png}
			\small{center control point}
			\end{center}
		\end{column}
		\uncover<2>{
			\begin{column}{0.4\textwidth}	
				\begin{multline*}
				E = (b_{210} + b_{120} + b_{021} \\ 
				+ b_{012} + b_{102} + b_{201}) / 6,
				\end{multline*}
				\begin{equation*}
				\begin{aligned}
				V & = (P_1 + P_2 + P_3)/ 3, \\
				b_{111} & = E + (E - V) / 2
				\end{aligned}
				\end{equation*}
			\end{column}
		}
	\end{columns}
	\note[item]<1>{\textbf{[Laura]} Note that this is the result of the previous step -> now only center coefficient is left.}
	\note[item]<2>{\textbf{[Laura]} Average of the tangent coefficients plus  half the difference between the tangent and vertex coefficients.}
\end{frame}	

\begin{frame}\frametitle{Geometry - Result}
	\enhancement{Set result slide to plain}
	\begin{columns}
		\begin{column}{0.6\textwidth}
			\begin{center}
			\includegraphics[width=\textwidth]{img/1_single/geometry_4.png}
			\end{center}	
		\end{column}
	\end{columns}
	\note<1>{\textbf{[Laura]} Results}
\end{frame}

\begin{frame}\frametitle{Overview}
	\begin{columns}
		\begin{column}{0.9\textwidth}
			\begin{center}
				\includegraphics[width=\textwidth]{./img/1_single/recap_geomToShading.png}
			\end{center}		
		\end{column}
	\end{columns}
	\note<1>{\textbf{[Rick]} Overview -> how to get from this to shading. Sample/subdivide with formula on following slide.}
\end{frame}	

\begin{frame}\frametitle{Cubic B\'ezier patch}
	\begin{columns}
		\begin{column}{0.5\textwidth}
			\includegraphics[width=\textwidth]{img/1_single/cubicPatch.png}
		\end{column}
		\begin{column}{0.5\textwidth}
			% \begin{equation*}
				% b: \mathbb{R}^2 \rightarrow \mathbb{R}^3 
				% \text{, for } w = 1 - u - v, u, v \text{, } w \geq 0
			% \end{equation*}
			\begin{equation*}
				w = 1 - u - v
			\end{equation*}
			\begin{equation*}
				u,\, v,\, w \geq 0
			\end{equation*}
			\begin{equation*}
				b(u,v) = \sum\limits_{i+j+k=3} b_{ijk} \frac{3!}{i!j!k!} u^i v^j w^k
				% & = b_{300} w^3 + b_{030} u^3 + b_{003} v^3 \\
				% & + b_{210} 3 w^2 u + b_{120} 3 w u^2 + b_{201} 3 w^2 v\\
				% & + b_{021} 3 u^2 v + b_{102} 3 w v^2 + b_{012} 3 u v^2\\
				% & + b_{111} 6 w u v.
			\end{equation*}
		\end{column}		
	\end{columns}
	\note<1>{\textbf{[Rick]} Very nice formula with a nice picture. Mention \textbf{Barycentric coordinates}}
	\note{u, v, w are a convex combination}
\end{frame}

%!TEX root = presentation.tex
\begin{frame}\frametitle{Overview}
	% \todo[inline]{The steps. Recap of everything construct geometry and normals and evaluate less (low lod) or more points (high lod)}
	\begin{columns}
		\begin{column}{0.9\textwidth}
			\begin{center}	
				\includegraphics[width=\textwidth]{./img/1_single/recap_inputToNormals.png}
			\end{center}		
		\end{column}
	\end{columns}
	\note<1>{\textbf{[Name]} From the primitive normals the the PN triangle normals}
\end{frame}	

\begin{frame}\frametitle{Normals}
\enhancement{emphasize normals more}
	\begin{columns}
		\begin{column}{0.6\textwidth}
			\begin{center}
				\includegraphics[width=\textwidth]{img/1_single/inputPrimitive_emphNormal.png}
				\small{input primitive}
			\end{center}
		\end{column}
	\end{columns}
	\note<1>{\textbf{[Name]} Recap input primitive and with emphasis on the normals.}
\end{frame}

\begin{frame}\frametitle{Normals - theory}
	\begin{columns}
		\begin{column}{0.6\textwidth}
			\begin{center}
				\includegraphics[width=\textwidth]{img/1_single/linearVsQuadraticNormals_linear.png}
				\small{linear}
			\end{center}	
		\end{column}
	\end{columns}
	\pause
	\vspace{0.8cm}
	\begin{columns}
		\begin{column}{0.6\textwidth}
			\begin{center}
				\includegraphics[width=\textwidth]{img/1_single/linearVsQuadraticNormals_quadratic.png}
				\small{quadratic}
			\end{center}	
		\end{column}
	\end{columns}
	\note<1>{\textbf{[Name]} Stress that there is a need to capture the cubic bezier curve (inflection points) and that this cannot be }
	\note<2>{\textbf{[Name]} Quadratic does capture inflection points. Trade off between performance and result (maybe?)}
\end{frame}

\begin{frame}\frametitle{Normals - example}
	\begin{columns}
		\begin{column}{0.4\textwidth}
		\begin{center}
				\includegraphics[width=\textwidth]{img/1_single/linearlyVaryingNormals.png}
				\small{linear}
			\end{center}	
		\end{column}
		\begin{column}{0.4\textwidth}
		\begin{center}
				\includegraphics[width=\textwidth]{img/1_single/quadriticallyVaryingNormals.png}
				\small{quadratic}
			\end{center}	
		\end{column}
	\end{columns}
	\note<1>{\textbf{[Name]} Look how pretty.}
\end{frame}


\begin{frame}\frametitle{Normals - theory}
	\begin{columns}
		\begin{column}{0.6\textwidth}
			\begin{center}
				\includegraphics[width=\textwidth]{img/1_single/computingNormals.png}
			\end{center}
		\end{column}
	\end{columns}
	
	\begin{columns}
		\begin{column}{0.4\textwidth}
			\uncover<2>{
				\begin{equation*}
				\begin{aligned}
					v_{ij}  & = 2\frac{(P_j - P_i) \cdot (N_i + N_j)}{(P_j - P_i) \cdot (P_j - P_i)} \in \mathbb{R} \\
					h_{110} & = N_1 + N2 - v_{12}(P_2 - P_1)\\
					n_{110} & =	h_{110} / ||h_{110}||
				\end{aligned}
				\end{equation*}
			}
		\end{column}
	\end{columns}
	\note{\textbf{[Name]} Formula in words: reflect the averaged normal (average of N1 and N2) on the plane orthogonal/perpendicular the the edge at the mid point.}
\end{frame}

\begin{frame}
	\frametitle{Normals - result}
	\enhancement{Set result slide to plain}
	\begin{columns}
		\begin{column}{0.6\textwidth}
			\begin{center}
				\includegraphics[width=\textwidth]{img/1_single/normals.png}
			\end{center}	
		\end{column}
	\end{columns}
	\note<1>{\textbf{[Name]} Result}
\end{frame}

% & = n_{200} w^2 + n_{020} u^2 + n_{002} v^2\\
% 		& + n_{110} w u + n_{011} u v + n_{101} w v

\begin{frame}
	\frametitle{Level Of Detail}
	\begin{columns}
		\begin{column}[b]{0.22\textwidth}
			\begin{center}
				\includegraphics[width=\textwidth]{./img/1_single/lod_lod0.png}
				\small{0}
			\end{center}	
		\end{column}
		\begin{column}[b]{0.22\textwidth}
			\begin{center}
				\includegraphics[width=\textwidth]{./img/1_single/lod_lod1.png}	
				\small{1}
			\end{center}	
		\end{column}
		\begin{column}[b]{0.22\textwidth}
			\begin{center}
				\includegraphics[width=\textwidth]{./img/1_single/lod_lod2.png}	
				\small{2}
			\end{center}	
		\end{column}
		\begin{column}[b]{0.22\textwidth}
			\begin{center}
				\includegraphics[width=\textwidth]{./img/1_single/lod_lod3.png}	
				\small{3}
			\end{center}	
		\end{column}
	\end{columns}
	\note<1>{\textbf{[Name]} Level of detail -> subdivision -> how many triangles go through to the next shaders.}	
\end{frame}

\begin{frame}\frametitle{Level Of Detail}
\begin{columns}
		\begin{column}[b]{0.6\textwidth}
			\begin{center}
				\includegraphics[width=\textwidth]{./img/1_single/lodExpanation.png}
				\small{$(u,\, v,\, w)$}
			\end{center}	
		\end{column}
	\end{columns}
\end{frame}

\begin{frame}\frametitle{Overview}
	% \todo[inline]{The steps. Recap of everything construct geometry and normals and evaluate less (low lod) or more points (high lod)}
	\begin{columns}
		\begin{column}{0.9\textwidth}
			\begin{center}
				\includegraphics[width=\textwidth]{./img/1_single/recap_overview.png}
			\end{center}		
		\end{column}
	\end{columns}
	\note<1>{\textbf{[Name]} Shading out of the scope of this presentation}
	\note<1>{\textbf{[Name]} Why quadartic patch for normals, why cubic patch for geometry}
\end{frame}	
