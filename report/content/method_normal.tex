%!TEX root = ../main.tex

\subsection{Normal component}\label{ss:normal_component}
% What?
In this section we discuss the parametrization of the normals. How the computed normals are actually a kind of `fake' normals as apposed to real normals, which are the normals evaluated the actual cubic B\`ezier surface. How both types of normals are constructed, is discussed in their similarly named subsections. 

Additionally as with the geometry of the PN triangle we again only have the information from the input primitive available i.e. the vertices positions and normals (see figure \ref{fig:method:input_primitive}).

\todo[inline]{Explain difference between fake and real normals.}

\todo[inline]{Explain structure of this section}

\subsubsection{Fake normals}
\label{sss:method:normals:fakeNormals}
	\todo[inline]{We express normal component as a `quadratic' patch}
	\todo[inline]{Discuss parametrization of `quadratic' patch}
	\todo[inline]{Discuss the construction of the control points for the `quadratic' patch}

\subsubsection{Real normals}
\label{sss:method:normals:realNormals}
	\todo[inline]{Discuss how to compute the real normals given the geometric component}