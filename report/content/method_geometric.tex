%!TEX root = ../main.tex

\subsection{Geometric component}
\label{ss:geometric_component}

As may have become clear from the previous section, the only available information for the construction of the geometric component (and normal component) are the three vertices that define a single primitive. The vertices contain the vertex \textit{xyz} coordinates together with a unique normal per vertex i.e. the normals of each vertex is the same for every primitive using that vertex. Figure \ref{fig:method:input_primitive} contains an illustration of ``'the input primitive'' for which we will henceforward will omit the explicit reference. 

The geometric component is in fact a cubic bezier triangle, which is defined as follows: 

\begin{equation}\label{eq:method:cubic_bezier_patch}
	a
\end{equation}

\todo[inline]{We express geometric compinent as a cubic patch}
\todo[inline]{Discuss parametrization of cubic patch}
\todo[inline]{Discuss the construction of the control points for the cubic patch}