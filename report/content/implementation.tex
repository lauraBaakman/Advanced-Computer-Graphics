%!TEX root = ../main.tex

\section{Implementation}
\label{s:implementation}
\rick{Nalezen:}

\begin{figure}
	\plaatje{Pas de kleuren aan, pas het font aan, maak de pipelines verticaal, laat de twee teseelations shaders zien, geef alle blokjes dezelfde stijl, met tikz maken?}
	\centering
	\begin{subfigure}{\columnwidth}
		\centering
		\includegraphics[width=\columnwidth]{content/img/implementation/pipeLineOld.png}
		\caption{The graphics pipeline of OpenGL version 1.3 (2001).}
		\label{fig:implementation:pipeline:old}
	\end{subfigure}
	\begin{subfigure}{\columnwidth}
		\centering
		\includegraphics[width=\columnwidth]{content/img/implementation/pipeLineNew.png}
		\caption{The graphics pipeline of OpenGL version 4.1 (2010).}
		\label{fig:implementation:pipeline:new}
	\end{subfigure}	
	\caption{A schematic overview of the programmable parts of the \subref{fig:implementation:pipeline:old} 1.3 and \subref{fig:implementation:pipeline:new} 4.1 OpenGL graphics pipelines. Each rounded rectangle represents one programmable shader.}
	\label{fig:implementation:pipeline}
\end{figure}

\Cref{fig:implementation:pipeline} presents a schematic overview of the OpenGL time at the time \citeauthor{vlachos2001curved} was published and a recent OpenGL version. The images show that since \citeyear{vlachos2001curved} three new shaders have been introduced into the pipeline, namely the tessellation control, tessellation evaluation and the geometry shader. 

% Geometry
% Input:
% Output:

% Tesselation control
% Invoked for
The tessellation control shader (TCS) is invoked once for each vertex in the output patch, a triangle in our case. 
% What does it do
In the TCS one can compute additional information about the input patch, e.g. for example control points. The primary function of the tessellation control shader is the setting of both the inner and the outer tessellation levels.
% Output:
The tessellation control shader passes additional information about the patch along the the tessellation evaluation shader and tells the tessellation primitive generator (TPG) how many primitives to generate.

% Tesselation evaluation
% Invoked for
% Input:
% What does it do
% Output:



\todo[inline]{Really short overview of the used openGL pipeline}
\todo[inline]{Explain structure of this section}

\subsection{Geometric component}
\todo[inline]{How do we compute the geometric component int he 2015 OpenGL pipeline}

\subsection{Fake normals}
\todo[inline]{How do we compute the fake normals int he 2015 OpenGL pipeline}

\subsection{Real normals}
\todo[inline]{How do we compute the real normals int he 2015 OpenGL pipeline}