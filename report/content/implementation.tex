%!TEX root = ../main.tex

\section{Implementation}
\label{s:implementation}
\rick{Nalezen:}

\begin{figure}
	\plaatje{Pas de kleuren aan, pas het font aan, maak de pipelines verticaal, laat de twee teseelations shaders zien, geef alle blokjes dezelfde stijl, met tikz maken?}
	\centering
	\begin{subfigure}{\columnwidth}
		\centering
		\includegraphics[width=\columnwidth]{content/img/implementation/pipeLineOld.png}
		\caption{The graphics pipeline of OpenGL version 1.3 (2001).}
		\label{fig:implementation:pipeline:old}
	\end{subfigure}
	\begin{subfigure}{\columnwidth}
		\centering
		\includegraphics[width=\columnwidth]{content/img/implementation/pipeLineNew.png}
		\caption{The graphics pipeline of OpenGL version 4.1 (2010).}
		\label{fig:implementation:pipeline:new}
	\end{subfigure}	
	\caption{A schematic overview of the programmable parts of the \subref{fig:implementation:pipeline:old} 1.3 and \subref{fig:implementation:pipeline:new} 4.1 OpenGL graphics pipelines. Each rounded rectangle represents one programmable shader.}
	\label{fig:implementation:pipeline}
\end{figure}

A schematic overview of both the OpenGL pipeline at the time of the publication of the paper and the introduces PN-triangles and the current pipeline are presented in \cref{fig:implementation:pipeline}. These images show that since \citeyear{vlachos2001curved} three new shaders have been introduced into the pipeline, namely the tessellation control, tessellation evaluation and the geometry shader. 

Since only the tessellation shaders are relevant for point-normal triangles we discuss those in more detail in section \ref{ss:implementation:pipeline}. Section \ref{ss:implementation:tcs} through section \ref{ss:implementation:tes} present our implementation of point-normal triangles per discussed tessellation stage. 

\subsection{The OpenGL Pipeline}
\label{ss:implementation:pipeline}

	\begin{figure}
		\centering
		\includegraphics[width=\columnwidth]{content/img/implementation/tesselationPipeline.png}
		\caption{An overview of the tessellation part of the OpenGL render pipeline. Note that in the case of point-normal triangles the input patches, generated primitives and output patches are triangles, not quads. Illustration taken form \textcite{wolff2013opengl}.}
		\label{fig:implementation:tessellationPipeline}
	\end{figure}

	% Tesselation:
	A schematic overview of tessellation in OpenGL is presented \Cref{fig:implementation:tessellationPipeline}.
	% Tesselation control
	% Invoked for
	The tessellation control shader (TCS) is invoked once for each vertex in the output patch, a triangle in our case. 
	% What does it do
	The primary function of the tessellation control shader is the setting of both the inner and the outer tessellation levels. Additionally the TCS can compute additional information about the vertices in the input patch. 
	% Output:
	As can been seen in \cref{fig:implementation:tessellationPipeline} the TCS passes additional information about that patch along to the tessellation evaluation shader (TES) via the output patch. The tessellation levels are sent to the tessellation primitive generator (TPG) \cite{wolff2013opengl}.

	The TPG generates a number of new primitives based on the tessellation levels. How these primitives are generated depends on the selected edge tessellation spacing. Equal spacing results in evenly divided edges, whereas fractional spacing allows for two shorter edges in the subdivision for more stable interpolation under changing tessellation levels \cite{wolff2013opengl,openGL41Core}.

	% Tesselation evaluation
	% Invoked for
	The tessellation evaluation shader (TES) is executed for each vertex that is generated by the TPG. 
	% Input:
	It receives the coordinates of this vertex in parameter space and the output of the TCS, as is illustrated in \cref{fig:implementation:tessellationPipeline}. 
	% What does it do
	Based on this information it computes the position of the current vertex. 
	% Output:
	This shader is passed to the next shader in the pipeline, either the geometry shader or the fragment shader, see \cref{fig:implementation:pipeline:new}.

\subsection{Tesselation Control Shader}
\label{ss:implementation:tcs}
	% Geometry

	% Fake normals

	% Real normals

\subsection{Tessellation Primitive Generator}
\label{ss:implementation:tpg}
	% Geometry

	% Fake normals

	% Real normals

\subsection{Tessellation Evaluation Shader}
\label{ss:implementation:tes}
	% Geometry

	% Fake normals

	% Real normals