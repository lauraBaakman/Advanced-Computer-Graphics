%!TEX root = ../main.tex


\section{Introduction}
\label{s:introduction}

% \todo[inline]{Context: Why PN triangles: fast smooth real-time rendering}
\textcite{vlachos2001curved} introduced curved point-normal triangles as a method to improve the visual quality of triangular meshes in real-time entertainment. A major advantage of this method at the time of introduction (\citeyear{vlachos2001curved}) is that it is a fast method to improve the visual quality of the rendered objects.

% \todo[inline]{Context: global idea of PN triangles}
If PN triangles are used, each triangle of the mesh is replaced by a curved shape, that can be subdivided into a user selected number of flat smaller triangles. Each point-normal triangle consists of a geometric and a normal component. These components are respectively defined as a cubic and a quadratic Bézier patch. The coefficient of the patches are based on the points and the normals of the corresponding triangle in the mesh. 

\iftoggle{PHONG}{
	\future{Context: global idea of phong tesselation}
	\future{Context: relationship between phong and pn}	
}


% \todo[inline]{Application Domain: useful for real time rendering, not for CAD, explain why not....}
Although curved point-normal triangles improve the visual appeal of a rendered mesh via``smoother silhouettes and better shading" \cite{vlachos2001curved} and are fast enough for real-time rendering, their application domain is restricted to entertainment. They should not be used in, for example computer-aided design. Since the rendering gives the model a visual smoothness that is not present in the actual mesh. 

% \todo[inline]{Required Background knowledge: nothing comes to mind, currently}

% \todo[inline]{Explain structure of report}
Section \ref{s:problem} introduces the problems we discuss, after that we treat the methods used to analyze the introduced problems in section \ref{s:method}. Section \ref{s:implementation} reviews our implementation of the discussed methods. We present and discuss our results in section \ref{s:results}. Finally section \ref{s:conclusion} presents our conclusions. 


