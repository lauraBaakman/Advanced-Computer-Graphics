%!TEX root = ../main.tex

\begin{abstract}
	\citeauthor{vlachos2001curved} introduced point-normal triangles in \citeyear{vlachos2001curved} to improve the visual quality of triangle meshes. The method was designed to be a fast, without the need to change the in original mesh. Since then a lot of changes have been made to the OpenGL pipeline, making it possible to render with point-normal triangles evaluated on the GPU. 

	% Wat gaan we doen % Wat is eruit gekomen: CPU/GPU
	We discuss how one can implemented point-normal triangles on the GPU and discuss the differences with the original CPU implementation. We have found that there are probably no differences.

	% Wat gaan we doen % Wat is eruit gekomen: normals
	Further more we also consider the use of normals based on the geometry, `real' normals, instead of the quadratically varying normals, `fake' normals, proposed by \citeauthor{vlachos2001curved}. We have concluded that although the resulting normal fields are the same the `fake' normals are computationally less expensive, and thus preferred.
	~\\

	\begin{classification}
		% Take a look at: http://academia.stackexchange.com/questions/15252/is-the-new-acm-2012-taxonomy-usable-in-use
		% \ccsdesc[500]{Computing methodologies~Parametric curve and surface models}
		\CCScat{Computer Graphics}{I.3.3}{Shape modeling}{Parametric curve and surface modeling}
	\end{classification}
\end{abstract}