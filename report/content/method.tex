%!TEX root = ../main.tex

\section{Method}
\label{s:method}

% \todo[inline]{Global idea of PN triangles, lit less global than in introduction}
The main goal of PN triangles is to improve the visual quality of rendered models, compare the models in \cref{fig:preamble:teaser}, and to do this on resource-limited hardware environments where e.g. no information about neighboring triangles can be accumulated. Or more precisely, the goal of point-normal triangles is ``to soften triangle creases and improve the visual appeal by generating smoother silhouettes and better shading" \cite{vlachos2001curved}. Other than the visual improvements PN triangles provide, \citeauthor{vlachos2001curved} mention the following benefits:

\begin{enumerate}
 	\item 
 		Point-normal triangle construction is \textit{compatible} with the existing API data structures i.e. vertex arrays together with triangle index streams are used, and the triangles arrive in an unpredictable order.
 	\item 
 		The models are \textit{backward compatible} with hardware that does not support point-normal triangles, with minimal or no changes needed to existing models.
 	\item 
 		No setup of the application, API, or hardware driver is needed. Specifically hardware should not be able to provide random access to neighboring primitives. Consequently the only possible communicated information between primitives are provided by using shared normals at the vertices. Which does restrict the models to be rendered somewhat, as discussed in section \ref{sss:method:geometric:properties}.
 	\item 
 		Point-normal triangles are applicable to \textit{arbitrary topology}.
 	\item 
 		PN-triangle rendering is done \textit{fast} and via \textit{simple implementation} in hardware on the CPU in 2001. At the time of writing even faster executing is possible by using programmable tessellation shaders on the GPU.
 \end{enumerate} 

In the remaining part of this section we discuss the construction of point-normal triangles conceptually as well as mathematically. As mentioned in the introduction, A PN triangle is split in two different components: we refer to section \ref{ss:geometric_component} for a discussion on the geometric component and to section \ref{ss:normal_component} for the review of the normal component.

\begin{figure}
	\centering
	\includegraphics[width=0.45\textwidth]{./content/img/method/input.png}
	\caption{An input primitive as used by point-normal triangle consists of three vertices and three vertex normals. Note that only the information from a single primitive is used during the construction of both the geometric and the normal component. The labels of the vertices and normals coincide with those used in the equations in this section.}
	\label{fig:method:input_primitive}
\end{figure}

%!TEX root = ../main.tex

\subsection{Geometric component}
\label{ss:geometric_component}
Let us emphasize again that the only used information for the construction of the geometric and normal component are the three vertices with associated normals, that define a single primitive. The vertices contain the vertex \textit{xyz}-coordinates together with a unique shared normal per vertex, i.e., the normal of each vertex is the same for every primitive using that vertex. Figure \ref{fig:method:input_primitive} shows an illustration of an input primitive. Each point-normal triangle construction starts with an input primitive of this form.

\begin{figure}
	\centering
	\includegraphics[width=0.45\textwidth]{./content/img/method/geometry.png}
	\caption{The geometric component of a point-normal triangle, i.e. the control net of a cubic Bézier triangle.}
	\label{fig:method:control_net}
\end{figure}
%
% ## Geometric component defined by triangluar cubic Bezier patch.
%
In the next subsection we discuss the definition of the geometry of a point-normal triangle. In subsection \crefs{sss:control_point_construction} we discuss the construction of the control points that define the geometry. 

\subsubsection{Basic form}
The geometric component of a point-normal triangle is defined as a triangular cubic Bézier patch. Such a patch $b(u,v)$ is defined as follows:
%
\begin{align}
\noalign{$b(u,v): \quad R^2 \mapsto R^3,\quad$ for $w = 1 - u - v, \quad u, v, w \geq 0$}
\begin{split}\label{eq:method:cubic_bezier_patch}
    b(u,v) ={}& \sum_{i + j + k = 3} b_{ijk}\frac{3!}{i!j!k!} u^i v^j w^k\\
      	   ={}& b_{300}w^3 + b_{030}u^3 + b_{003}v^3\\
      	    {}& + b_{210}3w^3 + b_{120}3wu^2 + b_{201}3w^2v\\
      	    {}& + b_{021}3u^2v + b_{102}2wv^2 + b_{012}3uv^2\\
      	    {}& + b_{111}6wuv.
\end{split}
\end{align}
%
The $b_{ijk}$ parameters in equation \ref{eq:method:cubic_bezier_patch} are the control points of the patch, also called coefficients. In \cref{fig:method:control_net} the visualization of the network of these control points is shown. We group the coefficients in three different groups, as their construction differs:
%
\begin{align*}
	\text{vertex coefficients: } {}&  b_{300},\ b_{030},\ b_{003} \\
	\text{tangent coefficients: } {}&  b_{210},\ b_{120},\ b_{021},\ b_{012},\ b_{102},\ b_{201}\\
	\text{center coefficient: }   {}&  b_{111}\\
\end{align*}

The formula in \eqref{eq:method:cubic_bezier_patch} can be used to evaluate any point parameterized by the barycentric coordinates $(u,v,w)$, on the patch. This is used in the sub-triangulation stage of the rendering process. In this stage the cubic Bézier surface is approximated by using a number of smaller flat triangles; these flat triangles are the triangles that are actually rendered. The number of sub-triangles is determined by the level of detail (lod). For the original sub-triangulation we refer the reader to \textcite{vlachos2001curved}, as this is where the implementation presented here deviates from the original. Details about this are provided in \crefs{s:implementation}.

As stated above the definition of the geometry of a point-normal triangle is a cubic Bézier patch. This degree of patches is a trade-off between simplicity, visual performance, and computational cost. Quadratic patches do not provide the same modeling range of a surface as a cubic patch, \citeauthor{boubekeur2008phong} \cite{boubekeur2008phong}. For example, a cubic representation is necessary to capture inflections implied by the triangle position and normal data. \citeauthor{vlachos2001curved} state that there is no additional data to suggest a higher degree is needed, and that therefore they settled on the form of $b(u,v)$ as presented in \crefe{eq:method:cubic_bezier_patch}.

\subsubsection{Coefficients} \label{sss:control_point_construction}
This section discusses how the `curved' control net geometry, (see \cref{fig:method:control_net}) is calculated from the flat input primitive shown in \cref{fig:method:input_primitive}. The input primitive provides the positions $P_1, P_2, P_3 \in \Real^3$ and normals $N_1, N_2, N_3 \in \Real^3$. The coefficients $b_{ijk}$ are computed as follows:
%
\begin{enumerate}[label=(\roman*)]
	\item 
		Initially, the coefficients $b_{ijk}$ are spread uniformly, i.e., the intermediate position of $b_{ijk}$ is calculated using the formula $(i P_i + j P_2 + kP_3) / 3$. 
	\item 
		The intermediate positions of the vertex coefficients are the same as their final positions, see equation \eqref{eq:method:vertex_coefficients}. Their intermediate position places them at the vertices of the input triangle, and this is where they are required to stay to keep the mesh watertight.
	\item 
		The tangent coefficients are placed at their final position by projecting the intermediate position into the tangent plane of the closest corner, see equation \eqref{eq:method:tangent_coefficients}. This is illustrated in \cref{fig:method:geometry_tangent_projection.png}.
	\item The center coefficient is moved to the average of the tangent coefficients plus $1/2$ times the distance it had to travel from its intermediate position, see equation \eqref{eq:method:center_coefficient}.
\end{enumerate}
%
\begin{figure}
	\centering
	\includegraphics[width=0.45\textwidth]{./content/img/method/geometry_tangent_projection.png}
	\caption{Projection of a tangent coefficient $b_{210}$ to the tangent plane of the closes corner $P_1$.}
	\label{fig:method:geometry_tangent_projection.png}
\end{figure}

The following set of formulas describe how the positions of the coefficients are calculated. For clarity, we group together the formulas in the same way as the coefficients. This gives the following formulas for the \textit{vertex coefficients}:
\begin{align}\label{eq:method:vertex_coefficients}
	b_{300} = P_1,\ b_{030} = P_2,\ b_{003} = P_3.
\end{align}

The tangent coefficients are given by the projection of a point $Q$ onto the plane defined by the normal $N$ of the point $P$. The projected point $Q'$ is then given by: $Q' = Q - wN$, where $w = (Q - P) \cdot N$. Using this, the following set of formulas give the positions of the \textit{tangent coefficients}:

\begin{align}\label{eq:method:tangent_coefficients}
	w_{ij} = {}& (P_j - P_i) \cdot N_i \in \Real \nonumber\\
	b_{210} = {}& (2 P_1 + P_2 - w_{12}N_1) / 3,\nonumber\\
	b_{120} = {}& (2 P_2 + P_1 - w_{21}N_2) / 3,\nonumber\\
	b_{021} = {}& (2 P_2 + P_3 - w_{23}N_2) / 3, \\
	b_{012} = {}& (2 P_3 + P_2 - w_{32}N_3) / 3,\nonumber\\
	b_{102} = {}& (2 P_3 + P_1 - w_{31}N_3) / 3,\nonumber\\
	b_{201} = {}& (2 P_1 + P_3 - w_{13}N_1) / 3. \nonumber
\end{align}

The last coefficient is the center coefficient which is, as stated before, moved to the average of the previous computed tangent coefficients plus $1/2$ times the distance it traveled from its intermediate location to that average position. The \textit{center coefficient}, is computed by:

\begin{align}\label{eq:method:center_coefficient}
	E = {}& (b_{210} + b_{120} + b_{021} \nonumber \\
		{}& + b_{012} + b_{102} + b_{201}) / 6, \nonumber\\
	V = {}& (P_1 + P_2 + P_3) / 3, \\
	b_{111} = {}& E + (E - V) / 2. \nonumber
\end{align}

Combining \eqref{eq:method:vertex_coefficients} through \eqref{eq:method:center_coefficient} transforms the input primitive (see \cref{fig:method:input_primitive}) to the control net shown in \cref{fig:method:control_net}.

\subsubsection{Properties}
\citeauthor{vlachos2001curved} have shown that PN triangles do not deviate to much from the original triangle. This is an important property because then the shape of the model is preserved an adjacent triangles do not interfere with each other. 

By demanding shared normals, i.e., one unique normal per vertex, the boundary between two point-normal triangles is generated by the same algorithm, thus the surface is water tight. Except at the corners, PN triangles do not usually join with tangent continuity \cite{vlachos2001curved}. \Cref{fig:method:cracks} illustrates what happens if normals are not shared.

\begin{figure}
	\ooit{Moeten dit plaatje eigenlijk maken met driehoeken..}
	\centering
	\includegraphics[width=0.4\columnwidth]{./content/img/method/cracks.png}
	\caption{Illustration of what would happen if one renders a model using point-normal triangles where vertices have different normals, depending on the associated faces. Image taken from \cite{mcdonald2010crack}.}
	\label{fig:method:cracks}
\end{figure}


%!TEX root = ../main.tex

\subsection{Normal component}\label{ss:normal_component}
% What?
In this section we discuss the parametrization of the normals. How the computed normals are actually a kind of `fake' normals. The real normals being the normals that are calculated using the geometry of the PN triangle thus a cubic B\`ezier surface. Section \ref{sss:method:normals:fakeNormals} and \ref{sss:method:normals:realNormals} provide the formulas and explanation for both types of normals and their construction. 

Additionally as with the geometry of the PN triangle we again only have the information from the input primitive available i.e. the vertices positions and normals (see figure \ref{fig:method:input_primitive}).

\begin{figure}
	\centering
	\includegraphics[width=0.45\textwidth]{./content/img/method/normals.png}
	\caption{The normal field of the PN triangle}
	\label{fig:method:normal_field}
\end{figure}

\subsubsection{Fake normals}\label{sss:method:normals:fakeNormals}
The `fake' normals following the definition from \citeauthor{vlachos2001curved} and are defined by the the quadratic function $n$: 
\begin{align}
\noalign{$n: \quad R^2 \mapsto R^3,\quad$ for $w = 1 - u - v, \quad u, v, w \geq 0$}
\begin{split}\label{eq:method:quadratic_normal_patch}
    n(u,v) ={}& \sum_{i + j + k = 2} n_{ijk}u^i v^j w^k,\\
      	   ={}& n_{200}w^2 + n_{020}u^2 + n_{002}v^2\\
      	    {}& + n_{110}wu + n_{011}uv + n_{101}wv\\
\end{split}
\end{align}
The coefficients of this quadratic `patch' are the normals shown in figure \ref{fig:method:normal_field}. The normals are computed for the point, halfway every edge. The reason why quadratically varying normals are used is, to capture the inflection points that are possible because of the use of the cubic patch for the geometry of the triangle. Figure \ref{fig:method:linear_vs_quadratically_varying} illustrates two cases: the top two images show a parabolic curve where both linear and quadratically varying normals perform the same. The more interesting case is the one illustrated by the two bottom images that show a cubic curve. We see that the linear varying normals do not capture the correct normal for this curve, but the quadratic varying normals do.

\begin{figure}
	\centering
	\includegraphics[width=0.45\textwidth]{./content/img/method/lin_vs_quad_varying_normals(inspiration).png}
	\caption{\textit{left:} linear varying normals. \textit{right:} quadratically varying normals. Taken from \citeauthor{van1997phong}}
	\label{fig:method:linear_vs_quadratically_varying}
\end{figure}
\todo[inline]{Discuss parametrization of `quadratic' patch}

Using the function $n$ the normal of any point parametrized by the barycentric coordinates $u, v$ can be calculated. To be able to do this the control points first need to be constructed. We group the coefficient into two groups: the vertex normals, $n_{200}$, $n_{020}$, and $n_{002}$; and the edge normals, $n_{110}$, $n_{011}$, and $n_{101}$. \todo{and some more text.}
\todo[inline]{Discuss the construction of the control points for the `quadratic' patch}

\subsubsection{Real normals}
\label{sss:method:normals:realNormals}
	\todo[inline]{Discuss how to compute the real normals given the geometric component}
	The real normals are computed using the points on the cubic B\`ezier patch. This can be done by taking the cross product of the the partial derivative with respect to $u$ and $v$.

