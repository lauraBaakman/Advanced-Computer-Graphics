%!TEX root = ../main.tex
\section{Problem Definition}
\label{s:problem}
We consider a number of subproblems associated with point-normal triangles.

\begin{enumerate}[label=(\roman*)]
	\item \label{it:problem:fakeNormals}
		% Waarom willen we het weten
		In the introduction we have mentioned that point-normal triangles are not suitable for rendering in computer aided designed software, due to the `fake' normal field, as the normal component of the PN triangle does not accurately represent the actual normals. Adapting the PN triangle algorithm to use `real' normals, expands its application domain to include rendering for CAD software.
		%
		% Wat willen we weten
		It is possible to compute the `real' normals of the vertices of the subdivided PN-triangle, using the method we present in section \ref{sss:method:normals:realNormals}. 
		%
		%Wat gaan we meten
		In this paper we compare both the computational and the visual performance of the `real' and `fake' normals. 
		% Wat verwachten we
		We do not expect a large difference in visual performance when `real' normals are used. Nor do we suppose that computing the normals based on the geometry will have a significant impact on the computational complexity of point-normal triangles.

\iftoggle{PHONG}{
	\item \label{it:problem:CompareWithPhongTesselation}
		\future{2: Compare PN triangles with phong tessellation on the following aspects: performance, continuity, visual. We try to answer the question: When use PN when Phong?}
}	

	\item \label{it:problem:GPUImplementation}
		%Waarom willen we het weten
		One major advantage of PN-triangles at the time of its introduction (\citeyear{vlachos2001curved}) was its speed. Since then geometry and tessellation shaders have been introduced into the OpenGL pipeline. 
		%
		% Wat willen we weten
		We investigate which changes have to be made to the point-normal triangles as proposed by \citeauthor{vlachos2001curved} to fit into a recent OpenGL pipeline (version 4.1).
		%
		% Hoe gaan we het meten
		In this paper we discuss the changes between the CPU implementation proposed by \citeauthor{vlachos2001curved} and our GPU implementation. 
		%
		% Wat verwachten we
		We expect that due to their locality point-normal triangles are extremely suited for computation on the GPU. Allowing us to tessellate triangular meshes faster than is possible on the CPU. 
\end{enumerate}