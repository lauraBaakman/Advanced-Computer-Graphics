%!TEX root = ../main.tex
\section{Problem Definition}
\rick{Nalezen:}
\label{s:problem}
We consider a number a number of sub problems associated with point-normal triangles.

\begin{enumerate}[label=(\roman*)]
	% \todo[inline]{1: Why use a fake normal field, the quadratic patch, if you can compute the normals of the vertices of the cubic patch.}
	\item \label{it:problem:fakeNormals}
	As mentioned in the introduction point-normal triangles use `fake' normals, i.e. the normal component of the PN triangle is in no way a reflection of the actual normals. In section \ref{sss:method:normals:realNormals} we show that it is possible to compute the `real' normals of the PN triangle based on its geometric component. We compare both computational and visual performance of these two methods. 

	\item \label{it:problem:CompareWithPhongTesselation}
	\future{2: Compare PN triangles with phong tesselation on the following aspects: performance, continuity, visual. We try to answer the question: When use PN when Phong?}

	\item \label{it:problem:GPUImplementation}
	% Wat willen we weten
	What changes have to be made to the point-normal triangles as proposed by \citeauthor{vlachos2001curved} to fit this rendering into a recent OpenGL version (4.1).  
	%
	% Hoe gaan we het testen
	In section \ref{s:implementation} we discuss the changes between the CPU implementation proposed by \citeauthor{vlachos2001curved} and our GPU implementation. 
\end{enumerate}